
\begin{table*}[tbp]
  \begin{centering}
  \begin{tabular}{l|lll}
  Processing Step/Run Condition& Reactor On & Reactor Off & Calibration \\ \hline
  Raw File Size (MB/run) & XX & XX &  XX \\
  Unpacked File Size (MB/run) & XX & XX &  XX \\
  Raw $\rightarrow$ Unpack processing time (CPU/hour/file)& XX & XX &  XX \\
  DetPulse File Size (MB/run) & XX & XX &  XX \\
  Unpack $\rightarrow$ DetPulse processing time (CPU/hour/file)& XX & XX &  XX \\
  PhysPulse File Size (MB/run) & XX & XX &  XX \\
  DetPulse $\rightarrow$ PhysPulse processing time (CPU/hour/file)& XX & XX &  XX \\
  \end{tabular}
  \caption[dataRates]
  {Data file sizes and processing time for three typical operating conditions. Sizes given are for a typical run length of XX min. Give time to process one reactor cycle and one year of data (for 7 HFIR cycles)}
  \label{tab:dataSize}
  \end{centering}
\end{table*}

Data is processed through the following steps in sequence:

\subsection{Raw data}

Raw data is output directly by the DAQ in a compressed binary format. One file exists for each digitizer board per run duration, which is typically ten minutes.

\subsection{Unpacked data}

The Unpacking step combines the raw data files from the multiple digitizer boards into a single file and converts from the compressed binary format of the raw data into a ROOT TTree.
The fundamental information, i.e. the digitizer waveforms, remains the same.
Thus, this step does not involve any physical or data analysis processing and only is a different format of the original data.
A channel map between the physical hardware channels and their ``logical'' functions ({e.g.} PMT positions in the detector) is included in the unpacked ROOT file.

\subsection{DetPulse data}
Unpacked data is processed through a custom software utility called PulseCruncher.
PulseCruncher reads each digitized waveform and identifies signal pulses there.
The output of the PulseCruncher is a file containing DetPulses, each of which has the following attributes:
\begin{itemize}
\item Event number (from the WFD board trigger counter)
\item PMT that detected this pulse
\item Pulse area (in uncalibrated digitizer units)
\item Pulse arrival time at PMT (in ns from the run start)
\item Pulse height (in uncalibrated digitizer units)
\item The waveform baseline that was subtracted from the raw data to measure the pulse height/area
\item Risetime of the pulse
\item Pulse-shape discrimination (PSD) parameter
\end{itemize}

In other words, this stage of processing converts digitized waveforms into a summary of the signal pulses in those waveforms, without applying any calibration.

\subsection{PhysPulse data}
A calibration is applied in this step, converting uncalibrated DetPulses to calibrated PhysPulses.
The calibration is applied using a database storing the interpreted calibration results extracted from earlier data.
Applying the calibration combines information from both PMTs in a pulse's segment,
	and so each PhysPulse is typically the combination of two DetPulses,
    	including information about the segment as a whole as well as the signal in each of the two PMTs.
Each PhysPulse contains the following information:
\begin{itemize}
\item Event number
\item Segment number
\item Pulse energy (in calibrated MeV\textsubscript{ee})
\item Pulse time (in ns from run start)
\item $\Delta t$, time difference between detection of this pulse in the two PMTs in this segment
\item Estimated number of photoelectrons detected each each PMT
\item Reconstructed position of the pulse along the segment axis
\item Pulse-shape discrimination parameter
\item The identified particle for this pulse.
	This indicates either a gamma/beta/muon or a neutron recoil, and separately whether the signal falls into the neutron-capture-on-\textsuperscript{6}Li region in PSD and energy. 
\end{itemize}



